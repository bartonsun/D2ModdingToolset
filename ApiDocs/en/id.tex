\subsection{Id}
\label{Id}
Represents object identifier. Identifiers used to search scenario objects.
\subsubsection{Properties}
\begin{center}
\begin{tabularx}{\linewidth}{| l | X |}
\hline
\textbf{Name} & \textbf{Description} \\
\hline
value & Returns integer representation of id. Can be used as Lua table key for best performance.\\
\hline
typeIndex & Returns identified object index among the same type of scenario objects (units, stacks, items, etc.). Can be used as Lua table key for best performance.\\
\hline
type & Returns \hyperref[IdType]{type} of identifier. Identifier type can help to distinguish one object from another\\
\hline
\end{tabularx}
\end{center}
\subsubsection{Methods}
\begin{center}
\begin{tabularx}{\linewidth}{| l | X |}
\hline
\textbf{Name} & \textbf{Description} \\
\hline
Id.new('S143KC0001') & Creates id from string\\
\hline
Id.emptyId() & Returns empty identifier\\
\hline
tostring(id) & Converts id to string\\
\hline
Id.summonId(position) & Creates special id for summoning units in battle using specified position in group. Position in group should be in \texttt{[0 : 5]} range.\\
\hline
\end{tabularx}
\end{center}
%\paragraph{Examples:}
%\begin{center}
%\begin{lstlisting}[language=Lua]
%\end{lstlisting}
%\end{center}