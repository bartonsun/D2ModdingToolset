\subsection{Scenario}
\label{Scenario}
Represents scenario map with all its objects and state
\subsubsection{Properties}
\begin{center}
\begin{tabularx}{\linewidth}{| l | X |}
\hline
\textbf{Name} & \textbf{Description} \\
\hline
name & Returns scenario name or empty string if scenario is unnamed\\
\hline
description & Returns scenario description or empty string if scenario has no description\\
\hline
author & Returns scenario author or empty string if no author specified\\
\hline
seed & Returns scenario initial seed used by random generator\\
\hline
day & Returns number of current day in game\\
\hline
size & Returns scenario map size\\
\hline
difficulty & Returns scenario map \hyperref[Difficulty]{difficulty} selected by player\\
\hline
variables & Returns \hyperref[ScenarioVariables]{scenario variables}. If scenario has no variables defined, returns \texttt{nil}\\
\hline
diplomacy & Returns object that holds \hyperref[Diplomacy]{diplomacy} relations between races. Fully loaded scenario always have diplomacy relations. During scenario loading this property can return \texttt{nil}\\
\hline
\end{tabularx}
\end{center}

\subsubsection{Methods}
\begin{center}
\begin{tabularx}{\linewidth}{| l | X |}
\hline
\textbf{Name} & \textbf{Description} \\
\hline
getLocation('S143LO0001') & Searches for \hyperref[Location]{location} by id string or \hyperref[Id]{id}, returns \texttt{nil} if not found\\
getLocation(Id.new('S143LO0001')) & \\
\hline
getPlayer('S143PL0001') & Searches for \hyperref[Player]{player} by id string or \hyperref[Id]{id}, returns \texttt{nil} if not found\\
getPlayer(Id.new('S143PL0001')) & \\
\hline
getUnit('S143UN0007') & Searches for \hyperref[Unit]{unit} by id string or \hyperref[Id]{id}, returns \texttt{nil} if not found\\
getUnit(Id.new('S143UN0007')) & \\
\hline
getItem('S143IM0001') & Searches for \hyperref[Item]{item} by id string or \hyperref[Id]{id}, returns \texttt{nil} if not found\\
getItem(Id.new('S143IM0001')) & \\
\hline
getTile(3, 5) & Searches for \hyperref[Tile]{tile} by pair of coordinates or \hyperref[Point]{point}, returns \texttt{nil} if not found\\
getTile(Point.new(3, 5)) &\\
\hline
getStack(10, 15) & Searches for \hyperref[Stack]{stack} by pair of coordinates, \hyperref[Point]{point}, id string or \hyperref[Id]{id}, returns \texttt{nil} if not found\\
getStack(Point.new(10, 15)) &\\
getStack('S143KC0005') &\\
getStack(Id.new('S143KC0005')) &\\
\hline
getFort(10, 15) & Searches for \hyperref[Fort]{fort} by pair of coordinates, \hyperref[Point]{point}, id string or \hyperref[Id]{id}, returns \texttt{nil} if not found\\
getFort(Point.new(10, 15)) &\\
getFort('S143FT0005') &\\
getFort(Id.new('S143FT0005')) &\\
\hline
getRuin(10, 15) & Searches for \hyperref[Ruin]{ruin} by pair of coordinates, \hyperref[Point]{point}, id string or \hyperref[Id]{id}, returns \texttt{nil} if not found\\
getRuin(Point.new(10, 15)) &\\
getRuin('S143RU0000') &\\
getRuin(Id.new('S143RU0000')) &\\
\hline
getRod(10, 15) & Searches for \hyperref[Rod]{rod} by pair of coordinates, \hyperref[Point]{point}, id string or \hyperref[Id]{id}, returns \texttt{nil} if not found\\
getRod(Point.new(10, 15)) &\\
getRod('S143RD0003') &\\
getRod(Id.new('S143RD0003')) &\\
\hline
getCrystal(10, 15) & Searches for \hyperref[Crystal]{crystal} by pair of coordinates, \hyperref[Point]{point}, id string or \hyperref[Id]{id}, returns \texttt{nil} if not found\\
getCrystal(Point.new(10, 15)) &\\
getCrystal('S143CR0004') &\\
getCrystal(Id.new('S143CR0004')) &\\
\hline
getMerchant(10, 15) & Searches for \hyperref[Merchant]{merchant} by pair of coordinates, \hyperref[Point]{point}, id string or \hyperref[Id]{id}, returns \texttt{nil} if not found\\
getMerchant(Point.new(10, 15)) &\\
getMerchant('S143SI0002') &\\
getMerchant(Id.new('S143SI0002')) &\\
\hline
getMercenary(10, 15) & Searches for \hyperref[Mercenary]{mercenary camp} by pair of coordinates, \hyperref[Point]{point}, id string or \hyperref[Id]{id}, returns \texttt{nil} if not found\\
getMercenary(Point.new(10, 15)) &\\
getMercenary('S143SI0002') &\\
getMercenary(Id.new('S143SI0002')) &\\
\hline
getTrainer(10, 15) & Searches for \hyperref[Trainer]{trainer} by pair of coordinates, \hyperref[Point]{point}, id string or \hyperref[Id]{id}, returns \texttt{nil} if not found\\
getTrainer(Point.new(10, 15)) &\\
getTrainer('S143SI0002') &\\
getTrainer(Id.new('S143SI0002')) &\\
\hline
getMarket(10, 15) & Searches for \hyperref[ResourceMarket]{market} by pair of coordinates, \hyperref[Point]{point}, id string or \hyperref[Id]{id}, returns \texttt{nil} if not found\\
getMarket(Point.new(10, 15)) &\\
getMarket('S143SI0002') &\\
getMarket(Id.new('S143SI0002')) &\\
\hline
\end{tabularx}
\end{center}
\begin{center}
\begin{tabularx}{\linewidth}{| l | X |}
\hline
\textbf{Name} & \textbf{Description} \\
\hline
findStackByUnit(unit) & Searches for \hyperref[Stack]{stack} that has specified \hyperref[Unit]{unit} among all the stacks in the whole scenario.\\
findStackByUnit('S143UN0007') & You can also use unit id string or \hyperref[Id]{id}. Returns \texttt{nil} if not found.\\
findStackByUnit(Id.new('S143UN0007')) & Note that \textbf{this search is heavy in terms of performance}, so you probably want to minimize excessive calls and use variables to store its results.\\
\hline
findFortByUnit(unit) & Searches for \hyperref[Fort]{fort} that has specified \hyperref[Unit]{unit} in its garrison among all the forts in the whole scenario. Only garrison units are counted, visiting stack is ignored.\\
findFortByUnit('S143UN0007') & You can also use unit id string or \hyperref[Id]{id}. Returns \texttt{nil} if not found.\\
findFortByUnit(Id.new('S143UN0007')) & Note that \textbf{this search is heavy in terms of performance}, so you probably want to minimize excessive calls and use variables to store its results.\\
\hline
findRuinByUnit(unit) & Searches for \hyperref[Ruin]{ruin} that has specified \hyperref[Unit]{unit} among all the forts in the whole scenario.\\
findRuinByUnit('S143UN0007') & You can also use unit id string or \hyperref[Id]{id}. Returns \texttt{nil} if not found.\\
findRuinByUnit(Id.new('S143UN0007')) & Note that \textbf{this search is heavy in terms of performance}, so you probably want to minimize excessive calls and use variables to store its results.\\
\hline
forEachStack(f) & Searches for every \hyperref[Stack]{stack} on a map and calls specified function on it\\
\hline
forEachLocation(f) & Searches for every \hyperref[Location]{location} on a map and calls specified function on it\\
\hline
forEachFort(f) & Searches for every \hyperref[Fort]{fort} on a map and calls specified function on it\\
\hline
forEachRuin(f) & Searches for every \hyperref[Ruin]{ruin} on a map and calls specified function on it\\
\hline
forEachRod(f) & Searches for every \hyperref[Rod]{rod} on a map and calls specified function on it\\
\hline
forEachPlayer(f) & Searches for every \hyperref[Player]{player} on a map and calls specified function on it\\
\hline
forEachUnit(f) & Searches for every \hyperref[Unit]{unit} on a map and calls specified function on it\\
\hline
forEachCrystal(f) & Searches for every \hyperref[Crystal]{crystal} on a map and calls specified function on it\\
\hline
forEachMerchant(f) & Searches for every \hyperref[Merchant]{merchant} on a map and calls specified function on it\\
\hline
forEachMercenary(f) & Searches for every \hyperref[Mercenary]{mercenary camp} on a map and calls specified function on it\\
\hline
forEachTrainer(f) & Searches for every \hyperref[Trainer]{trainer} on a map and calls specified function on it\\
\hline
forEachMarket(f) & Searches for every \hyperref[ResourceMarket]{market} on a map and calls specified function on it\\
\hline
\end{tabularx}
\end{center}