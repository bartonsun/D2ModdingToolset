\subsection{Point}
\label{Point}
Represents point in 2D space.
\subsubsection{Properties}
\begin{center}
\begin{tabularx}{\linewidth}{| l | X |}
\hline
\textbf{Name} & \textbf{Description} \\
\hline
x & Access x coordinate for reading and writing\\
\hline
y & Access y coordinate for reading and writing\\
\hline
\end{tabularx}
\end{center}
\subsubsection{Methods}
\begin{center}
\begin{tabularx}{\linewidth}{| l | X |}
\hline
\textbf{Name} & \textbf{Description} \\
\hline
Point.new() & Creates point with both coordinates set to 0\\
\hline
Point.new(x, y) & Creates point with specified coordinates\\
\hline
tostring(p) & Converts point to string \texttt{'(x, y)'}\\
\hline
\end{tabularx}
\end{center}
\subsubsection{Examples}
\begin{center}
\begin{lstlisting}[language=Lua]
-- x = 0, y = 0
local p = Point.new()
-- x = 1, y = 3
local p2 = Point.new(1, 3)
-- s == '(1, 3)'
local s = tostring(p2)
\end{lstlisting}
\end{center}